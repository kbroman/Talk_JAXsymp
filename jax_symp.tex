\documentclass[aspectratio=169,12pt,t]{beamer}
\usepackage{graphicx}
\setbeameroption{hide notes}
\setbeamertemplate{note page}[plain]
\usepackage{listings}

\input{header.tex}

%%%%%%%%%%%%%%%%%%%%%%%%%%%%%%%%%%%%%%%%%%%%%%%%%%%%%%%%%%%%%%%%%%%%%%
% end of header
%%%%%%%%%%%%%%%%%%%%%%%%%%%%%%%%%%%%%%%%%%%%%%%%%%%%%%%%%%%%%%%%%%%%%%

% title info
\title{Reproducible research}
\author{\href{https://kbroman.org}{Karl Broman}}
\institute{Biostatistics \& Medical Informatics, UW{\textendash}Madison}
\date{\href{https://kbroman.org}{\tt \scriptsize \color{foreground} kbroman.org}
\\[-4pt]
\href{https://github.com/kbroman}{\tt \scriptsize \color{foreground} github.com/kbroman}
\\[-4pt]
\href{https://rstats.me/@kbroman}{\tt \scriptsize \color{foreground} @kbroman@rstats.me}
\\[2pt]
\scriptsize {\lolit Slides:} \href{https://kbroman.org/Talk_JAXsymp}{\tt
  \color{foreground} kbroman.org/Talk\_JAXsymp}
}


\begin{document}

% title slide
{
\setbeamertemplate{footline}{} % no page number here
\frame{
  \titlepage

  \bigskip \bigskip
  \hfill
  \includegraphics[height=5mm]{Figs/cc-zero.png}

  \note{This is a much-shortened version of lecture of my usual
    lecture on ``steps to reproducible research'' (see
    {\tt https://github.com/kbroman/Talk\_ReproRes}).

    This is for a Jackson Lab symposium on 6 Nov 2025, part of the
    preparations to develop a course on omics analysis.

    Source: {\tt https://github.com/kbroman/Talk\_JAXsymp} \\
    These slides: {\tt https://kbroman.org/Talk\_JAXsymp} \\
    Slides with notes: {\tt https://kbroman.org/Talk\_JAXsymp/jax\_symp\_withnotes.pdf}

    By ``reproducible research,'' I'm referring to ``computational
    reproducibility,'' by which I mean that the data and code for a
    project are packaged together in a way that they can be handed to
    someone else, who can rerun the code and get the same
    results---the same figures and tables. This is surprisingly hard
    to do, and it's even more difficult in the context of a
    collaboration between two or more data analysts.
}
} }


\begin{frame}[fragile,c]{}

\begin{center}
\begin{minipage}[c]{9.3cm}
\begin{semiverbatim}
\lstset{basicstyle=\normalsize}
\begin{lstlisting}[linewidth=9.3cm]
 Karl -- this is very interesting,
 however you used an old version of
 the data (n=143 rather than n=226).

 I'm really sorry you did all that
 work on the incomplete dataset.

 Bruce
\end{lstlisting}
\end{semiverbatim}
\end{minipage}
\end{center}

\note{I'm an applied statistician; my goal is to help people make
  sense of their data. I have a lot of collaborators, and there's
  nothing I enjoy more than puzzling over their data. So I write a lot of
  reports, describing what I've done and what I've learned.

  This is an email I got from a collaborator,
  in response to an analysis report that I had sent him.
  It's always a bit of a shock to get an email like this: what have I
  done? Why am I working with the wrong data, and where is the right data?

  But what he didn't know is that by this point in my life, I'd
  adopted a reproducible workflow.
  Because I'd set things up carefully, I could just substitute in the
  newer dataset, type a single command (``{\tt make}'') to rerun the
  analyses, and get the revised report.

  This is a reproducibility success story. We all make mistakes, but
  if our projects are reproducible, we can nimbly recover from those
  mistakes.

  There is a second important lesson here: At the start of such
  reports, I always include a paragraph about our shared goals, along
  with some brief data summaries. By doing so, he immediately saw that
  I had an old version of the data. If I hadn't done so, we might
  never have discovered my error.
}
\end{frame}


\begin{frame}[c]{}
\centering
{\Large The results in Table 1 don't seem to \\[12pt]
correspond to those in Figure 2.}

\note{My computational life is not entirely rosy. This is the sort of
  email that will freak me out.}
\end{frame}


\begin{frame}[c]{}
\centerline{\Large Where did we get this data file?}

\note{Record the provenance of all data or metadata files.}
\end{frame}



\begin{frame}[c]{}
\centerline{\Large Why did I omit those samples?}

\note{I may decide to omit a few samples. Will I record {\nhilit why}
  I omitted those particular samples?}
\end{frame}



\begin{frame}[c]{}
\centerline{\Large Which image goes with which experiment?}

\note{For experimental biologists, it can be tricky to keep track of
  the vast set of images and experiments they perform.}
\end{frame}



\begin{frame}[c]{}
\centerline{\Large How did I make that figure?}

\note{Sometimes, in the midst of a bout of exploratory data analysis,
  I'll create some exciting graph and have a heck of a time
  reproducing it afterwards.}
\end{frame}


\begin{frame}[c]{}
\centerline{\Large In what order do I run these scripts?}

\note{Sometimes the process of data file manipulation and data
  cleaning gets spread across a bunch of scripts that need to be
  executed in a particular order. Will I record this information? Is
  it obvious what script does what?}
\end{frame}



\begin{frame}[c]{}
\centerline{\Large ``Your script is now giving an error."}

\note{It was working last week. Well, last month, at least.

How easy is it to go back through that script's history to see when
and why it stopped working?}
\end{frame}



\begin{frame}[c]{}
\centerline{\Large ``The attached is similar to the code we used."}

\note{From an email in response to my request for code used for a
  paper.}
\end{frame}



\begin{frame}[c]{Reproducible research}

\begin{quotation}
{\normalfont
  organize the data and code in a way \\[4pt]
  that you can hand them to someone else \\[4pt]
  and they can re-run the code \\[4pt]
  and get the same results \\[4pt]
  \quad (the same figures and tables)
}
\end{quotation}

\note{
  To reiterate my definition of reproducible research:
  it's about assembly and organizing the data and code
  so that they can be re-run to give the same results.
}
\end{frame}






\begin{frame}{Steps to reproducible research}

\bigskip

  \bi
\item Organize your data and code
\item Everything with a script
\item Automate the process
\item Turn scripts into reproducible reports
\item Turn repeated code into functions
\item Package functions for reuse
\item Use version control
\item License your software
  \ei


\bigskip \bigskip
\bigskip


\hfill {\tt \footnotesize \lolit \href{https://kbroman.org/steps2rr}{kbroman.org/steps2rr}}

\note{
  More than 10 years ago, while preparing for a workshop to develop a
  course on reproducible research, I thought through the steps one
  might follow, when transitioning from ``standard practice'' to a
  fully reproducible workflow. I created the website
  \href{https://kbroman.org/steps2rr}{\tt kbroman.org/steps2rr}.
}
\end{frame}




\begin{frame}{Additional considerations}

\bigskip

  \bi
  \item Arranging data within files
  \item Metadata
  \item Sharing data and code
  \item Software testing
  \item Capturing the software environment
  \item Containers
  \item Handling large-scale computations
  \item Coordinating with collaborators
  \ei


\note{
  There are a number of additional important things that I hadn't
  covered in \href{https://kbroman.org/steps2rr}{\tt kbroman.org/steps2rr}.
}
\end{frame}



\begin{frame}{Challenges in collaborations}

\bigskip

 \bi
\item Shared vision
\item Compromise
\item Coordination
\item Communication
\item Sharing code and data
\item Synchronization
  \ei

  \note{
    Collaboration also has challenges.

    Do you have a shared vision for the reproducibility of the
    project? You'll no doubt need to make some compromises about how
    things are done: you can't both just do things the way you've
    always done them. Careful coordination and regular communication
    are key.

    And then there are the technical challenges of how to share the
    code and data and make sure your two working projects remain in
    sync.

    In a sense, the reproducibility of a collaborative project is
    dependent on the weakest link. If one collaborator refuses to
    fully participate and share their work, the chain is broken.
}

\end{frame}



\begin{frame}{Tools}

\bigskip

  \bi
\item R, python
\item R Markdown, Quarto, Jupyter notebooks
\item GNU make, snakemake, targets
\item functions and packages
\item git and GitHub
\item renv, conda
\item docker
  \ei

\note{
  There are quite a lot of new tools to learn, when seeking to adopt a
  reproducible life.
}

\end{frame}




\begin{frame}{}

\bigskip
\bigskip
\bigskip
\bigskip

\begin{center}
\large
The most important tool is the {\hilit mindset},\\
when starting, that the end product \\
will be reproducible.
\end{center}

\hfill
{\lolit
{\textendash} Keith Baggerly
}

\bigskip \bigskip \bigskip

\onslide<2>{
\begin{center}
\large
The second-most important tool is {\hilit training}.
\end{center}
}

\note{So true. Desire for reproducibility is step one.

And I've long felt that the key need, in getting computational
scientists to adopt a reproducible workflow, is training. For the
most part, all of the software tools are available, but many people
haven't incorporated them into their daily work.
}
\end{frame}




\begin{frame}[c]{}

\Large

Slides: \href{https://kbroman.org/Talk_JAXsymp}{\tt kbroman.org/Talk\_JAXsymp} \quad
\includegraphics[height=5mm]{Figs/cc-zero.png}

\vspace{10mm}

\href{https://kbroman.org}{\tt kbroman.org}

\vspace{10mm}

\href{https://github.com/kbroman}{\tt github.com/kbroman}

\vspace{10mm}

\href{https://rstats.me/@kbroman}{\tt @kbroman@rstats.me}


\note{
  Here's where you can find me, as well as the slides for this talk.
}
\end{frame}




\end{document}
